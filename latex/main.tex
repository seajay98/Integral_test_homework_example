\documentclass[reqno,14pt]{amsart}
\usepackage{amsmath, amssymb, amsthm, mathtools}
\usepackage{graphicx}
\usepackage{enumitem}
\usepackage[letterpaper, margin=1in]{geometry}
\usepackage[usenames, dvipsnames]{xcolor}
\usepackage[colorlinks=true, linkcolor=Black, citecolor=Black, urlcolor=Black]{hyperref}
\usepackage[justification=centering]{caption}
\setlist{
  listparindent=\parindent,
  parsep=0pt,
}

\theoremstyle{plain}
\newtheorem{theorem}{Theorem}
\newtheorem{lemma}{Lemma}[section]
\newtheorem{proposition}[lemma]{Proposition}
\newtheorem{corollary}[lemma]{Corollary}
\newcommand{\e}{\epsilon}

\title[Estimating a Series with the Integral Test]{Estimating an Infinite Series Using the Integral Test: A Step-by-Step Example}
\author{Prepared by Charles Destefani}
\date{2022}

\begin{document}

\maketitle

\begin{abstract}
This short article walks through how to estimate the value of a convergent $p$-series using the integral test, step by step. The example problem, taken from a Calculus II homework set, illustrates how to determine the number of terms needed to approximate an infinite series within a given error tolerance.
\end{abstract}

\section*{Homework 10.3, Question 11}

\textbf{Problem:} Estimate the value of $\displaystyle \sum_{n=1}^\infty \frac{1}{n^3}$ to within $0.009$ of its exact value.

\medskip

To start, we recognize that this is a convergent $p$-series because $p = 3$. But $p$-series do not have a formula for their exact sum, so we must find the value another way. The best approach is to use the \textbf{integral test}.

Before we can apply the integral test, we must first check that the function in question is continuous, positive, and decreasing. The function that corresponds to our series is:
\[
    f(x) = \frac{1}{x^3}, \quad \text{so} \quad \sum_{n=1}^\infty \frac{1}{n^3} \longrightarrow \int_1^\infty \frac{1}{x^3}\,dx.
\]
This function satisfies the criteria of the integral test, so we can use the integral of $f(x)$ to approximate the sum.

\medskip

We want to estimate the sum to be within $0.009$ of the true sum. In other words, the remainder term $R_n$ (the error between the $n$th partial sum and the true infinite sum) must satisfy
\[
    R_n = |S - S_n| \le 0.009.
\]
We use absolute value because we want the approximation to be that close in both directions, not just one. This can also be expressed as
\[
    -0.009 \le S - S_n \le 0.009,
\]
or equivalently,
\[
    S_n - 0.009 \le S \le S_n + 0.009.
\]
If we can find the smallest integer $n$ that makes this inequality true, then we will have found the number of terms required to achieve the desired accuracy. Our approximation for the sum will then be the $n$th partial sum $S_n$, since it lies in the middle of the interval.

\medskip

Next, we can simplify our analysis by using the fact that all terms in the series are positive. Because every term is positive, the sequence of partial sums is increasing, so $S_n \le S$. This means $S - S_n$ is never negative, and we can simplify our bounds to:
\[
    0 \le S - S_n \le 0.009,
\]
which implies
\[
    S_n \le S \le S_n + 0.009.
\]

\medskip

Now we turn to the integral test to find the smallest $n$ that satisfies these bounds. The integral test tells us that the remainder $R_n$ can be bounded by:
\[
    \int_{n+1}^\infty \frac{1}{x^3}\,dx \le R_n \le \int_n^\infty \frac{1}{x^3}\,dx.
\]
We can easily compute these integrals:
\[
    \int \frac{1}{x^3}\,dx = -\frac{1}{2x^2}.
\]
Applying the limits, we get:
\[
    \frac{1}{2(n+1)^2} \le R_n \le \frac{1}{2n^2}.
\]

\medskip

We now want $R_n \le 0.009$, so:
\[
    \frac{1}{2n^2} \le 0.009.
\]
Solving for $n$ gives:
\[
    n^2 \ge \frac{1}{2(0.009)} = 55.56 \quad \Rightarrow \quad n \ge 7.45.
\]
Thus, if $n = 8$ (or any larger integer), we will satisfy the error condition.

\medskip

Finally, our approximation for $S$ will be the 8th partial sum:
\[
    S_8 = \sum_{n=1}^8 \frac{1}{n^3} = 1 + \frac{1}{2^3} + \frac{1}{3^3} + \frac{1}{4^3} + \frac{1}{5^3} + \frac{1}{6^3} + \frac{1}{7^3} + \frac{1}{8^3}.
\]
Evaluating this gives:
\[
    S_8 = 1.195.
\]

\medskip

Therefore, our estimate of the infinite sum is:
\[
    S \approx 1.195 \quad \text{and} \quad |S - S_8| < 0.009.
\]
\end{document}